\documentclass[]{article}
\usepackage{lmodern}
\usepackage{amssymb,amsmath}
\usepackage{ifxetex,ifluatex}
\usepackage{fixltx2e} % provides \textsubscript
\ifnum 0\ifxetex 1\fi\ifluatex 1\fi=0 % if pdftex
  \usepackage[T1]{fontenc}
  \usepackage[utf8]{inputenc}
\else % if luatex or xelatex
  \ifxetex
    \usepackage{mathspec}
  \else
    \usepackage{fontspec}
  \fi
  \defaultfontfeatures{Ligatures=TeX,Scale=MatchLowercase}
\fi
% use upquote if available, for straight quotes in verbatim environments
\IfFileExists{upquote.sty}{\usepackage{upquote}}{}
% use microtype if available
\IfFileExists{microtype.sty}{%
\usepackage{microtype}
\UseMicrotypeSet[protrusion]{basicmath} % disable protrusion for tt fonts
}{}
\usepackage[margin=1in]{geometry}
\usepackage{hyperref}
\PassOptionsToPackage{usenames,dvipsnames}{color} % color is loaded by hyperref
\hypersetup{unicode=true,
            pdftitle={MODIStsp: A Tool for Automatic Preprocessing of MODIS Time Series - v1.3.3},
            pdfauthor={Lorenzo Busetto (busetto.l@irea.cnr.it), Luigi Ranghetti (ranghetti.l@irea.cnr.it)},
            colorlinks=true,
            linkcolor=blue,
            citecolor=Blue,
            urlcolor=blue,
            breaklinks=true}
\urlstyle{same}  % don't use monospace font for urls
\usepackage{color}
\usepackage{fancyvrb}
\newcommand{\VerbBar}{|}
\newcommand{\VERB}{\Verb[commandchars=\\\{\}]}
\DefineVerbatimEnvironment{Highlighting}{Verbatim}{commandchars=\\\{\}}
% Add ',fontsize=\small' for more characters per line
\usepackage{framed}
\definecolor{shadecolor}{RGB}{248,248,248}
\newenvironment{Shaded}{\begin{snugshade}}{\end{snugshade}}
\newcommand{\KeywordTok}[1]{\textcolor[rgb]{0.13,0.29,0.53}{\textbf{#1}}}
\newcommand{\DataTypeTok}[1]{\textcolor[rgb]{0.13,0.29,0.53}{#1}}
\newcommand{\DecValTok}[1]{\textcolor[rgb]{0.00,0.00,0.81}{#1}}
\newcommand{\BaseNTok}[1]{\textcolor[rgb]{0.00,0.00,0.81}{#1}}
\newcommand{\FloatTok}[1]{\textcolor[rgb]{0.00,0.00,0.81}{#1}}
\newcommand{\ConstantTok}[1]{\textcolor[rgb]{0.00,0.00,0.00}{#1}}
\newcommand{\CharTok}[1]{\textcolor[rgb]{0.31,0.60,0.02}{#1}}
\newcommand{\SpecialCharTok}[1]{\textcolor[rgb]{0.00,0.00,0.00}{#1}}
\newcommand{\StringTok}[1]{\textcolor[rgb]{0.31,0.60,0.02}{#1}}
\newcommand{\VerbatimStringTok}[1]{\textcolor[rgb]{0.31,0.60,0.02}{#1}}
\newcommand{\SpecialStringTok}[1]{\textcolor[rgb]{0.31,0.60,0.02}{#1}}
\newcommand{\ImportTok}[1]{#1}
\newcommand{\CommentTok}[1]{\textcolor[rgb]{0.56,0.35,0.01}{\textit{#1}}}
\newcommand{\DocumentationTok}[1]{\textcolor[rgb]{0.56,0.35,0.01}{\textbf{\textit{#1}}}}
\newcommand{\AnnotationTok}[1]{\textcolor[rgb]{0.56,0.35,0.01}{\textbf{\textit{#1}}}}
\newcommand{\CommentVarTok}[1]{\textcolor[rgb]{0.56,0.35,0.01}{\textbf{\textit{#1}}}}
\newcommand{\OtherTok}[1]{\textcolor[rgb]{0.56,0.35,0.01}{#1}}
\newcommand{\FunctionTok}[1]{\textcolor[rgb]{0.00,0.00,0.00}{#1}}
\newcommand{\VariableTok}[1]{\textcolor[rgb]{0.00,0.00,0.00}{#1}}
\newcommand{\ControlFlowTok}[1]{\textcolor[rgb]{0.13,0.29,0.53}{\textbf{#1}}}
\newcommand{\OperatorTok}[1]{\textcolor[rgb]{0.81,0.36,0.00}{\textbf{#1}}}
\newcommand{\BuiltInTok}[1]{#1}
\newcommand{\ExtensionTok}[1]{#1}
\newcommand{\PreprocessorTok}[1]{\textcolor[rgb]{0.56,0.35,0.01}{\textit{#1}}}
\newcommand{\AttributeTok}[1]{\textcolor[rgb]{0.77,0.63,0.00}{#1}}
\newcommand{\RegionMarkerTok}[1]{#1}
\newcommand{\InformationTok}[1]{\textcolor[rgb]{0.56,0.35,0.01}{\textbf{\textit{#1}}}}
\newcommand{\WarningTok}[1]{\textcolor[rgb]{0.56,0.35,0.01}{\textbf{\textit{#1}}}}
\newcommand{\AlertTok}[1]{\textcolor[rgb]{0.94,0.16,0.16}{#1}}
\newcommand{\ErrorTok}[1]{\textcolor[rgb]{0.64,0.00,0.00}{\textbf{#1}}}
\newcommand{\NormalTok}[1]{#1}
\usepackage{longtable,booktabs}
\usepackage{graphicx,grffile}
\makeatletter
\def\maxwidth{\ifdim\Gin@nat@width>\linewidth\linewidth\else\Gin@nat@width\fi}
\def\maxheight{\ifdim\Gin@nat@height>\textheight\textheight\else\Gin@nat@height\fi}
\makeatother
% Scale images if necessary, so that they will not overflow the page
% margins by default, and it is still possible to overwrite the defaults
% using explicit options in \includegraphics[width, height, ...]{}
\setkeys{Gin}{width=\maxwidth,height=\maxheight,keepaspectratio}
\IfFileExists{parskip.sty}{%
\usepackage{parskip}
}{% else
\setlength{\parindent}{0pt}
\setlength{\parskip}{6pt plus 2pt minus 1pt}
}
\setlength{\emergencystretch}{3em}  % prevent overfull lines
\providecommand{\tightlist}{%
  \setlength{\itemsep}{0pt}\setlength{\parskip}{0pt}}
\setcounter{secnumdepth}{5}
% Redefines (sub)paragraphs to behave more like sections
\ifx\paragraph\undefined\else
\let\oldparagraph\paragraph
\renewcommand{\paragraph}[1]{\oldparagraph{#1}\mbox{}}
\fi
\ifx\subparagraph\undefined\else
\let\oldsubparagraph\subparagraph
\renewcommand{\subparagraph}[1]{\oldsubparagraph{#1}\mbox{}}
\fi

%%% Use protect on footnotes to avoid problems with footnotes in titles
\let\rmarkdownfootnote\footnote%
\def\footnote{\protect\rmarkdownfootnote}

%%% Change title format to be more compact
\usepackage{titling}

% Create subtitle command for use in maketitle
\newcommand{\subtitle}[1]{
  \posttitle{
    \begin{center}\large#1\end{center}
    }
}

\setlength{\droptitle}{-2em}
  \title{MODIStsp: A Tool for Automatic Preprocessing of MODIS Time Series -
v1.3.3}
  \pretitle{\vspace{\droptitle}\centering\huge}
  \posttitle{\par}
  \author{Lorenzo Busetto
(\href{mailto:lbusett@gmail.com}{busetto.l@irea.cnr.it}), Luigi
Ranghetti
(\href{mailto:ranghetti.l@irea.cnr.it}{\nolinkurl{ranghetti.l@irea.cnr.it}})}
  \preauthor{\centering\large\emph}
  \postauthor{\par}
  \predate{\centering\large\emph}
  \postdate{\par}
  \date{2018-03-11}


\begin{document}
\maketitle

{
\hypersetup{linkcolor=black}
\setcounter{tocdepth}{2}
\tableofcontents
}
\section{Introduction}\label{introduction}

MODIStsp is a novel ``R'' package allowing to automatize the creation of
time series of rasters derived from MODIS Land Products data. It allows
to perform several preprocessing steps on MODIS data available within a
given time period.

Development of MODIStsp started from modifications of the ModisDownload
``R'' script by Thomas Hengl (2010), and successive adaptations by Babak
Naimi (2014). The basic functionalities for download and preprocessing
of MODIS datasets provided by these scripts were gradually incremented
with the aim of:

\begin{itemize}
\tightlist
\item
  developing a stand-alone application allowing to perform several
  preprocessing steps (e.g., download, mosaicing, reprojection and
  resize) on all available MODIS land products by exploiting a powerful
  and user-friendly GUI front-end;
\item
  allowing the creation of time series of both MODIS original layers and
  additional Quality Indicators (e.g., data acquisition quality,
  cloud/snow presence, algorithm used for data production, etc. )
  extracted from the aggregated bit-field QA layers;
\item
  allowing the automatic calculation and creation of time series of
  several additional Spectral Indexes starting form MODIS surface
  reflectance products.
\end{itemize}

All processing parameters can be easily set with a user-friendly GUI,
although non-interactive execution exploiting a previously created
Options File is possible. Stand-alone execution outside an ``R''
environment is also possible, allowing to use scheduled execution of
MODIStsp to automatically update time series related to a MODIS product
and extent whenever a new image is available.

Required MODIS HDF files are automatically downloaded from NASA servers
and resized, reprojected, resampled and processed according to user's
choices. For each desired output layer, outputs are saved as single-band
rasters corresponding to each acquisition date available for the
selected MODIS product within the specified time period. ``R''
\emph{RasterStack} objects with temporal information as well as Virtual
raster files (GDAL vrt and ENVI META files) facilitating access to the
entire time series can be also created.

\section{Installation}\label{installation}

\texttt{MODIStsp} requires \href{https://cran.r-project.org}{R} v
\textgreater{}= 3.2.1 and \href{http://www.gdal.org}{GDAL} (Geospatial
Data Abstraction Library) v \textgreater{}= 1.11.1 \textbf{with support
for HDF4 raster format} to be installed in your system. Brief
instructions for installing R and GDAL can be found
\protect\hyperlink{installing-r-and-gdal}{HERE}.

\subsection{On Windows}\label{on-windows}

You can install the stable version of \texttt{MODIStsp}, from CRAN:

\texttt{install.packages("MODIStsp")}

, or the development version (containing the latest improvements and bug
fixes):

\begin{Shaded}
\begin{Highlighting}[]
\KeywordTok{library}\NormalTok{(devtools)}
\KeywordTok{install_github}\NormalTok{(}\StringTok{"lbusett/MODIStsp"}\NormalTok{)}
\end{Highlighting}
\end{Shaded}

Note that \textbf{if the \texttt{GTK+} library is not already installed
on your system, installation may fail}. In that case, please install and
load the \texttt{gWidgetsRGtk2} library beforehand:

\begin{Shaded}
\begin{Highlighting}[]
\KeywordTok{install.packages}\NormalTok{(}\StringTok{"gWidgetsRGtk2"}\NormalTok{)}
\KeywordTok{library}\NormalTok{(gWidgetsRGtk2)}
\end{Highlighting}
\end{Shaded}

Upon loading \texttt{gWidgetsRGtk2} , an error window will probably
appear. This signals that library ``GTK+'' is not yet installed on your
system or is not on your PATH. To install it press ``OK''. A new window
dialog window will appear, asking if you want to install ``GTK+''.
Select ``Install GTK'' and then ``OK'' . Windows will download and
install the GTK+ library. When it finishes, the RSession should be
restarted and you should be ready to go !

In case RStudio does not automatically restart or continuously asks to
install GTK+ again, kill it form ``Task Manager'' (or restart the R
session from RStudio ``Session'' menu), reload RStudio and the try to
reload \texttt{gWidgetsRGtk2}. If it loads correctly, you should be
ready to go.

If it still fails, try downloading the GTK+ bundle from:

\url{http://ftp.gnome.org/pub/gnome/binaries/win64/gtk+/2.22/gtk+-bundle_2.22.1-20101229_win64.zip}
(OR
\url{http://ftp.gnome.org/pub/gnome/binaries/win32/gtk+/2.22/gtk+-bundle_2.22.1-20101227_win32.zip}
if on Win32)

, unzip the archive on a folder of your choice (e.g.,
\texttt{C:\textbackslash{}\textbackslash{}Program\ Files\textbackslash{}\textbackslash{}GTK+}),
then add the path to its ``bin'' subfolder (e.g.,
\texttt{C:\textbackslash{}\textbackslash{}Program\ Files\textbackslash{}\textbackslash{}GTK+\textbackslash{}\textbackslash{}bin\textbackslash{}\textbackslash{}}
to your system PATH environment variable.

Restart your system and try loading again \texttt{gWidgetsRGtk2}: if it
loads ok, you should be ready to install \texttt{MODIStsp}

\subsection{On Linux systems}\label{on-linux-systems}

To install \texttt{MODIStsp} on Linux, you have to first install the
following required dependencies:

\begin{itemize}
\tightlist
\item
  \texttt{Cairo} \textgreater{}= 1.0.0, \texttt{ATK} \textgreater{}=
  1.10.0, \texttt{Pango} \textgreater{}= 1.10.0, \texttt{GTK+}
  \textgreater{}= 2.8.0, \texttt{GLib} \textgreater{}= 2.8.0 (required
  by package \texttt{RGtk2})
\item
  \texttt{Curl} (required by package \texttt{curl})
\item
  \texttt{GDAL} \textgreater{}= 1.6.3, \texttt{PROJ.4} \textgreater{}=
  4.4.9 (required by package \texttt{rgdal})
\end{itemize}

On \emph{Debian and Ubuntu-based} systems, to install those packages
open a terminal and type:

\begin{Shaded}
\begin{Highlighting}[]
\FunctionTok{sudo}\NormalTok{ apt-get install r-cran-cairodevice r-cran-rgtk2 libcairo2-dev libatk1.0-dev libpango1.0-dev }
\ExtensionTok{libgtk2.0-dev}\NormalTok{ libglib2.0-dev libcurl4-openssl-dev libgdal-dev libproj-dev}
\end{Highlighting}
\end{Shaded}

On \emph{rpm-base systems}, to install packages open a terminal and
type:

\begin{Shaded}
\begin{Highlighting}[]
\FunctionTok{sudo}\NormalTok{ yum install libcairo2-devel libatk1.0-devel libpango1.0-devel gtk2 gtk2-devel }
\ExtensionTok{glib2-devel}\NormalTok{ libcurl-devel gdal-devel proj proj-devel proj-epsg proj-nad}
\end{Highlighting}
\end{Shaded}

Then, you can install the stable version of MODIStsp from CRAN:

\begin{Shaded}
\begin{Highlighting}[]
\KeywordTok{install.packages}\NormalTok{(}\StringTok{"MODIStsp"}\NormalTok{)}
\end{Highlighting}
\end{Shaded}

, or the development version (containing the latest improvements and bug
fixes) from github;

\begin{Shaded}
\begin{Highlighting}[]
\KeywordTok{library}\NormalTok{(devtools)}
\KeywordTok{install_github}\NormalTok{(}\StringTok{"lbusett/MODIStsp"}\NormalTok{)}
\end{Highlighting}
\end{Shaded}

\subsection{On Mac OS}\label{on-mac-os}

\textbf{NOTE}: The following installation notes should be valid for
MODIStsp installation on R 3.4.0 and above with Mac OSX Sierra. They
were mainly taken (i.e., blatantly copied\ldots{}) from:
\url{https://zhiyzuo.github.io/installation-rattle/**}. Thanks to Zhiya
Zuo for providing this!

To properly install \texttt{MODIStsp} you will need to first install
package \texttt{RGTk2}. This is a somehow difficult operation. The
following instructions should help:

\textbf{1. Check your Mac OS X version and update if necessary: }

Enter the following command in terminal to check your macOS version.
Expected output is as below the dashed line ---.

\begin{verbatim}
~$ sw_vers  
------------------------  
ProductName:    Mac OS X  
ProductVersion: 10.12.6  
BuildVersion:   16G29  
\end{verbatim}

If your system is above 10.11, continue. Otherwise, upgrade it to
Sierra.

Install homebrew if you do not have it already installed. homebrew is a
very convenient package manager for macOS. To do so, open a terminal,
copy the following command in it and hit Enter:

\begin{verbatim}
~$ /usr/bin/ruby -e "$(curl -fsSL https://raw.githubusercontent.com/Homebrew/install/master/install)"
\end{verbatim}

Follow the instructions to get brew ready. When inserting your password,
nothing will show up for security reasons. Just hit Enter when you are
finished.

When brew is finished, copy the following command in terminal and hit
Enter:

\begin{verbatim}
~$ touch ~/.bash_profile
~$ echo "export PATH=/usr/local/bin:$PATH
export PKG_CONFIG_PATH=/usr/local/lib/pkgconfig:/usr/local/lib/pkgconfig/gtk+-2.0.pc:/opt/X11/lib/pkgconfig" >> ~/.bash_profile
~$ source ~/.bash_profile
\end{verbatim}

\begin{enumerate}
\def\labelenumi{\arabic{enumi}.}
\setcounter{enumi}{1}
\tightlist
\item
  \textbf{Install the \texttt{cairo} library with x11 support}. To do
  so, you first have to change the way homebrew wants to install gtk+.
  In an editor, write:
\end{enumerate}

\begin{verbatim}
~$ brew edit gtk+
\end{verbatim}

A text editor will open. Look in the file, and find a section that
begins with ``def install''. Substitute the current \texttt{args} with
the following text:

\begin{verbatim}
def install
 args = [
         "--disable-dependency-tracking",
         "--disable-silent-rules",
         "--prefix=#{prefix}",
         "--disable-glibtest",
         "--enable-introspection=yes",
         # "--disable-visibility",
         # "--with-gdktarget=quartz",
         "--with-gdktarget=x11",
         "--enable-x11-backend"
        ]
\end{verbatim}

Save the modified file using \texttt{ctrl+x\ ctrl+c}, followed by
\texttt{y} to quit emacs.

\begin{enumerate}
\def\labelenumi{\arabic{enumi}.}
\setcounter{enumi}{2}
\tightlist
\item
  \textbf{Install \texttt{gtk+}} by issuing this command:
\end{enumerate}

\begin{verbatim}
~$ brew install --build-from-source --verbose gtk+
\end{verbatim}

\begin{enumerate}
\def\labelenumi{\arabic{enumi}.}
\setcounter{enumi}{3}
\tightlist
\item
  \textbf{Update your path} so that \texttt{gtk+} is recognized, using:
\end{enumerate}

\begin{verbatim}
~$ export PKG_CONFIG_PATH=/usr/local/lib/pkgconfig:/usr/local/lib/pkgconfig/gtk+-2.0.pc:/opt/X11/lib/pkgconfig
\end{verbatim}

\begin{enumerate}
\def\labelenumi{\arabic{enumi}.}
\setcounter{enumi}{4}
\tightlist
\item
  \textbf{Install \texttt{RGtk2} from source}:
\end{enumerate}

\begin{itemize}
\item
  \textbf{Download the newest source file for RGtk2} from
  \url{https://cran.r-project.org/web/packages/RGtk2/index.html}.
\item
  Assuming that the path to this file is \textasciitilde{}/Downloads.
  Run the following in terminal (change the path if you did not download
  in \textasciitilde{}/Downloads):
\end{itemize}

\begin{verbatim}
~$ cd ~/Downloads
~/Downloads$ R CMD INSTALL RGtk2_2.20.33.tar.gz
\end{verbatim}

(Note that the name of the tar.gz file may vary depending on when you
downloaded the file).

\begin{enumerate}
\def\labelenumi{\arabic{enumi}.}
\setcounter{enumi}{5}
\tightlist
\item
  \textbf{Open R and run}:
\end{enumerate}

\begin{Shaded}
\begin{Highlighting}[]
\KeywordTok{library}\NormalTok{(RGtk2)}
\end{Highlighting}
\end{Shaded}

hopefully, \texttt{RGtk2} will load without errors! If so, you should be
ready to go, and you can:

\begin{enumerate}
\def\labelenumi{\arabic{enumi}.}
\setcounter{enumi}{6}
\tightlist
\item
  \textbf{Install MODIStsp} from CRAN:
\end{enumerate}

\begin{Shaded}
\begin{Highlighting}[]
\KeywordTok{install.packages}\NormalTok{(}\StringTok{"MODIStsp"}\NormalTok{)}
\KeywordTok{MODIStsp}\NormalTok{()}
\end{Highlighting}
\end{Shaded}

or the development version from GitHub:

\begin{Shaded}
\begin{Highlighting}[]
\KeywordTok{library}\NormalTok{(devtools) }
\KeywordTok{install_github}\NormalTok{(}\StringTok{"lbusett/MODIStsp"}\NormalTok{, }\DataTypeTok{ref =} \StringTok{"master"}\NormalTok{)}
\KeywordTok{MODIStsp}\NormalTok{()}
\end{Highlighting}
\end{Shaded}

Good luck!

\section{Running the tool in Interactive Mode: the MODIStsp
GUI}\label{running-the-tool-in-interactive-mode-the-modistsp-gui}

The easiest way to use \texttt{MODIStsp} is to use its powerful GUI
(Graphical User Interface) for selection of processing options, and then
run the processing.

To open the GUI, load the package and launch the MODIStsp function, with
no parameters:

\begin{Shaded}
\begin{Highlighting}[]
\KeywordTok{library}\NormalTok{(MODIStsp)}
\KeywordTok{MODIStsp}\NormalTok{()}
\end{Highlighting}
\end{Shaded}

This \textbf{opens a GUI} from which processing options can be specified
and eventually saved (or loaded) (Note 1: PCs with a small screen can
fail to visualize the whole GUI; in this case, the user can add scroll
bars with \texttt{MODIStsp(scrollWindow=TRUE)}); Note 2: At the first
execution of \texttt{MODIStsp}, a Welcome screen will appear, signaling
that \texttt{MODIStsp} is searching for a valid GDAL installation. Press
``OK'' and wait for GDAL to be found. If nothing happens for a long time
(e.g., several minutes), \texttt{MODIStsp} (and in particular the
gdalUtils package on which it relies) is not finding a valid GDAL
installation in the more common locations. To solve the problem: 1.
Ensure that GDAL is properly installed in your system 2. (On Windows) If
it is installed, verify that GDAL is in your system PATH. and that the
\emph{GDAL\_DATA} environment variable is correctly set (You can find
simple instructions
\href{http://gisforthought.com/setting-up-your-gdal-and-ogr-environmental-variables/}{HERE})
3. If nothing works, signal it in the
\href{https://github.com/lbusett/MODIStsp/issues}{issues} GitHub page of
\texttt{MODIStsp} and we'll try to help!

The GUI allows selecting all processing options required for the
creation of the desired MODIS time series. The main available processing
options are described in detail in the following.

\begin{center}\includegraphics{MODIStsp_files/figure-latex/GUIfig-1} \end{center}

\subsection{\texorpdfstring{\textbf{Selecting Processing
Parameters}}{Selecting Processing Parameters}}\label{selecting-processing-parameters}

\subsubsection{\texorpdfstring{\textbf{\emph{MODIS Product, Platform and
Layers:}}}{MODIS Product, Platform and Layers:}}\label{modis-product-platform-and-layers}

The top-most menus allow to specify details of the desired output time
series:

\begin{enumerate}
\def\labelenumi{\arabic{enumi}.}
\tightlist
\item
  \textbf{``Category''} and \textbf{``Product''}: Selects the MODIS
  product of interest;
\item
  \textbf{MODIS platform(s)}: Selects if only TERRA, only AQUA or Both
  MODIS platforms should be considered for download and creation of the
  time series;
\item
  \textbf{version}: Selects whether processing version 5 or 6 (when
  available) of MODIS products has to be processed
\end{enumerate}

After selecting the product and version, clicking the \textbf{``Change
Selection''} button opens the \emph{\emph{Select Processing Layers}} GUI
panel, from which the user \textbf{must} select which MODIS original
layers and/or derived Quality Indexes (QI) and Spectral Indexes (SI)
layers should be processed:

\begin{center}\includegraphics{MODIStsp_files/figure-latex/proc_layers-1} \end{center}

\begin{itemize}
\tightlist
\item
  The left-hand frame allows to select which \emph{original MODIS
  layers} should be processed.
\item
  The central frame allows to select which \emph{Quality Indicators
  should be extracted} from the original MODIS Quality Assurance layers.
\item
  For MODIS products containing surface reflectance data, the right-hand
  frame allows to select which additional \emph{Spectral Indexes should
  be computed}. The following commonly used Spectral Indexes are
  available for computation by default:
\end{itemize}

\textbf{Table II: List of default Spectral Indexes available in
MODIStsp}

\begin{longtable}[]{@{}ll@{}}
\toprule
Index Acronym & Index name and reference\tabularnewline
\midrule
\endhead
NDVI & Normalized Difference Vegetation Index (Rouse et al.
1973)\tabularnewline
EVI & Enhanced Vegetation Index (A. Huete et al. 2002)\tabularnewline
SR & Simple Ratio (Tucker 1979)\tabularnewline
NDFI & Normalized Difference Flood Index (Boschetti et al.
2014)\tabularnewline
NDII7 (NDWI) & Normalized Difference Infrared Index - Band 7 (Hunt and
Rock 1989)\tabularnewline
SAVI & Soil Adjusted Vegetation Index (A.R Huete 1988)\tabularnewline
NDSI & Normalized Difference Snow Index (Hall et al.
2002)\tabularnewline
NDII6 & Normalized Difference Infrared Index - band 6 (Hunt and Rock
1989)\tabularnewline
GNDVI & Green Normalized Difference Vegetation Index (Gitelson and
Merzlyak 1998)\tabularnewline
RGRI & Red Green Ratio Index (Gamon and Surfus 1999)\tabularnewline
GRVI & Green-Red ratio Vegetation Index (Tucker 1979)\tabularnewline
\bottomrule
\end{longtable}

You can however \textbf{specify other SIs to be computed without
modifying MODIStsp source code} by clicking on the \emph{\textbf{``Add
Custom Index''}} button, which allow to provide info related to the new
desired SI using a simple GUI interface.

\begin{center}\includegraphics{MODIStsp_files/figure-latex/indexfig-1} \end{center}

Provided information (e.g., correct band-names, computable formula,
etc\ldots{}) is automatically checked upon clicking ``Set New Index''.
On success, the new index is added in the list of available ones for all
products allowing its computation. Clicking ``Done !'' returns to the
main.

\textbf{NOTE} All custom defined indexes can be removed by using the
\texttt{MODIStsp\_resetindexes()} function

\subsubsection{\texorpdfstring{\textbf{\emph{Download
Method:}}}{Download Method:}}\label{download-method}

Select the method to be used for download. Available choices are:

\begin{enumerate}
\def\labelenumi{\arabic{enumi}.}
\item
  \textbf{http}: download through ftp from NASA lpdaac http archive
  (\url{http://e4ftl01.cr.usgs.gov}). This requires providing a user
  name and password, which can be obtained by registering an account at
  the address \url{https://urs.earthdata.nasa.gov/profile};
\item
  \textbf{ftp}: download from NASA ftp archive
  (\url{ftp://ladsweb.nascom.nasa.gov/});
\item
  \textbf{offline}: this option allows to process/reprocess HDF files
  already available on the user's PC without downloading from NASA --
  useful if the user already has an archive of HDF images, or to
  reprocess data already downloaded via MODIStsp to create time series
  for an additional layer (\emph{It is fundamental that the HDFs are
  those directly downloaded from NASA servers} ! (See
  \href{faq.html\#working-with-already-downloaded-hdf-files}{here} for
  additional details).
\end{enumerate}

Checking the \textbf{use\_aria2c} option allows to accelerate the
download from NASA archives. This requires however that that the
``aria2c'' software is installed in your system. To download and install
it, see: \href{aria2.github.io/}{https://aria2.github.io/}

 \emph{NOTE: The best performances are usually achieved using
\emph{http}, though that may vary depending on network infrastructure.}

\subsubsection{\texorpdfstring{\textbf{\emph{Processing
Period:}}}{Processing Period:}}\label{processing-period}

Specify the starting and ending dates to be considered for the creation
of the time in the series corresponding fields. Dates \textbf{must} be
provided in the \emph{yyyy--mm--dd} format (e.g., 2015-01-31)

The \textbf{Period} drop-down menu allows to choose between two options:

\begin{enumerate}
\def\labelenumi{\arabic{enumi}.}
\item
  \textbf{full}: all available images between the starting and ending
  dates are downloaded and processed;
\item
  \textbf{seasonal}: data is downloaded only for one part of the year,
  but for multiple years. For example, if the starting date is
  2005-03-01 and the ending is 2010-06-01, only the images of March,
  April and May for the years between 2005 and 2010 will be downloaded.
  This allows to easily process data concerning a particular season of
  interest.
\end{enumerate}

\subsubsection{\texorpdfstring{\textbf{\emph{Spatial
Extent:}}}{Spatial Extent:}}\label{spatial-extent}

Allows to define the area of interest for the processing. Two main
options are possible:

\begin{enumerate}
\def\labelenumi{\arabic{enumi}.}
\item
  \textbf{Select MODIS Tiles}: specify which MODIS tiles need to be
  processed either by:

  \begin{enumerate}
  \def\labelenumii{\alph{enumii}.}
  \tightlist
  \item
    Using the ``Start'' and ``End'' horizontal and vertical sliders in
    the \emph{Required MODIS Tiles} frame.\\
  \item
    pressing the \textbf{``Select on Map''} button. A map will open in a
    browser window, allowing interactive selection of the required tiles
  \end{enumerate}
\end{enumerate}

During processing, data from the different tiles is mosaiced, and a
single file covering the total area is produced for each acquisition
date

\begin{enumerate}
\def\labelenumi{\arabic{enumi}.}
\setcounter{enumi}{1}
\item
  \textbf{Define Custom Area}: specify a custom spatial extent for the
  desired outputs either by:

  \begin{enumerate}
  \def\labelenumii{\alph{enumii}.}
  \item
    Manually inserting the coordinates of the Upper Left and Lower Right
    corners of the area of interest in the \textbf{Bounding Box} frame.
    \emph{Coordinates of the corners must be provided in the coordinate
    system of the selected output projection};
  \item
    pressing the \textbf{``Load Extent from a Spatial File'' and
    selecting a raster or vector spatial file}. In this case, the
    bounding box of the selected file is retrieved, converted in the
    selected output projection, and shown in the ``Bounding Box'' frame.
    Required input MODIS tiles are also automatically retrieved from the
    output extent, and the tiles selection sliders modified accordingly.
  \item
    pressing the \textbf{``Select on Map''} button. A map will open in a
    browser window, allowing interactive selection of the spatial extent
    using the tools on the left.
  \end{enumerate}
\end{enumerate}

\textbf{Note: } pressing the ``show current extent'' will open a browser
window highlighting the currently selected spatial extent.

\subsubsection{\texorpdfstring{\emph{Reprojection and
Resize}}{Reprojection and Resize}}\label{reprojection-and-resize}

Specify the options to be used for reprojecting and resizing the MODIS
images.

\begin{itemize}
\item
  \textbf{``Output Projection''}: select either the Native MODIS
  projection (Default) or specify a user-defined one. To specify a user
  selected projection, select ``User Defined'' and then insert a valid
  ``Proj4'' string in the pop-up window. Validity of the Proj4 string is
  automatically checked, and error messages issued if the check fails;
\item
  \textbf{``Output Resolution''}, \textbf{``Pixel Size''} and
  \textbf{``Reprojection Method''}: specify whether output images should
  inherit their spatial resolution from the original MODIS files, or be
  resampled to a user-defined resolution. In the latter case, output
  spatial resolution must be specified in the measure units of the
  selected output projection. Resampling method can be chosen among
  ``Nearest Neighbour'' and ``Mode'' (Useful for down-sampling
  purposes). Other resampling methods (e.g., bilinear, cubic) are not
  currently supported since i) they cannot be used for resampling of
  categorical variables such as the QA and QI layers, and ii) using them
  on continuous variable (e.g., reflectance, VI values) without
  performing an a-priori data cleaning would risk to contaminate the
  values of high-quality observations with those of low-quality ones.
\end{itemize}

\subsubsection{\texorpdfstring{\emph{Output
Options}}{Output Options}}\label{output-options}

Several processing options can be set using check-boxes:

\begin{itemize}
\item
  \textbf{Output Files Format}: Two of the most commonly formats used in
  remote sensing applications are available at the moment: ENVI binary
  and GeoTiff. If GeoTiff is selected, the type of file compression can
  be also specified among ``None'', ``PACKBITS'', ``LZW'' and
  ``DEFLATE''.
\item
  \textbf{Save Time Series as}: Specify if virtual multitemporal files
  should be created. These virtual files allow access to the entire time
  series of images as a single file without the need of creating large
  multitemporal raster images. Available virtual files formats are ``R''
  rasterStacks, ENVI meta-files and GDAL ``vrt'' files. In particular,
  ``R'' rasterStacks may be useful in order to easily access the
  preprocessed MODIS data within ``R'' scripts (see also
  \url{http://lbusett.github.io/MODIStsp/articles/output.html}).
\item
  \textbf{Modify No Data}: Specify if NoData values of MODIS layers
  should be kept at their original values, or changed to those specified
  within the ``MODIStsp\_Products\_Opts'' XML file. By selecting ``Yes''
  in the ``Change Original NoData values'' check-box, NoData of outputs
  are set to the largest integer value possible for the data type of the
  processed layer (e.g., for 8-bit unsigned integer layers, NoData is
  set always to 255, for 16-bit signed integer layers to 32767, and for
  16-bit unsigned integer layers to 65535). Information about the new
  NoData values is stored both in the output rasters, and in the XML
  files associated with them. \textbf{NOTE:} Some MODIS layers have
  multiple NoData (a.k.a. \emph{fill}) values. if \emph{Modify No Data}
  is set to ``Yes'', \texttt{MODIStsp} will convert all \emph{fill}
  values to a common output NoData value!
\item
  \textbf{Apply Scale/Offset}: Specify if scale and offset values of the
  different MODIS layers should be applied. If selected, outputs are
  appropriately rescaled on the fly, and saved in the true ``measure
  units'' of the selected parameter (e.g., spectral indexes are saved as
  floating point values; Land Surface Temperature is saved in degrees
  Kelvin, etc.).
\end{itemize}

\subsubsection{\texorpdfstring{\textbf{Main MODIStsp Output
Folder}}{Main MODIStsp Output Folder}}\label{main-modistsp-output-folder}

Select the main folder where the pre-processed time series data will be
stored. All \texttt{MODIStsp} outputs \textbf{will be placed in specific
sub-folders of this main folder} (see
\url{http://lbusett.github.io/MODIStsp/articles/output.html} for details
on \texttt{MODIStsp} naming conventions)-.

The \textbf{``Reprocess Existing Data''} check-box allows to decide if
images already available should be reprocessed if a new run of MODIStsp
is launched with the same output folder. If set to ``No'', MODIStsp
skips dates for which output files following the MODIStsp naming
conventions are already present in the output folder. This allows to
incrementally extend MODIS time series without reprocessing already
available dates.

\subsubsection{\texorpdfstring{\emph{Folder for permanent storage of
original MODIS HDF
images}}{Folder for permanent storage of original MODIS HDF images}}\label{folder-for-permanent-storage-of-original-modis-hdf-images}

Select the folder where downloaded \textbf{original MODIS HDF files}
downloaded from NASA servers will be stored.

The \textbf{``delete original HDF files''} check-box allows also to
decide if the downloaded images should be deleted from the file system
at the end of the processing. To avoid accidental file deletion, this is
always set to ``No'' by default, and a warning is issued before
execution whenever the selection is changed to ``Yes''.

\subsection{\texorpdfstring{\textbf{Saving and Loading Processing
Options}}{Saving and Loading Processing Options}}\label{saving-and-loading-processing-options}

Specified processing parameters can be saved to a JSON file for later
use by clicking on the \emph{\textbf{Save Options}} button.

Previously saved options can be restored clicking on the
\emph{\textbf{Load Options}} button and navigating to the previously
saved JSON file.

 (Note that at launch, \textbf{\texttt{MODIStsp} \emph{always reloads
automatically the processing options used for its last successful run}}
.

\subsection{\texorpdfstring{\textbf{Starting the
processing}}{Starting the processing}}\label{starting-the-processing}

Once you are happy with your choices, click on \textbf{Start
Processing}. \texttt{MODIStsp} will start accessing NASA servers to
download and process the MODIS data corresponding to your choices.

For each date of the specified time period, \texttt{MODIStp} downloads
and preprocesses all hdf images required to cover the desired spatial
extent. Informative messages concerning the status of the processing are
provided on the console, as well as on a self-updating progress window.

The processed time series are saved in specific subfolders of the main
selected output folder, as explained in detail
\href{articles/output.html}{HERE}.

\section{Non-Interactive Execution from within
R}\label{non-interactive-execution-from-within-r}

\subsection{\texorpdfstring{Specifying a saved ``Options
file''}{Specifying a saved Options file}}\label{specifying-a-saved-options-file}

\texttt{MODIStsp} can be launched in non-interactive mode within an
\texttt{R} session by setting the optional \texttt{GUI} parameter to
FALSE, and the \texttt{Options\_File} parameter to the path of a
previously saved JSON Options file. This allows to exploit
\texttt{MODIStsp} functionalities within generic ``R'' processing
scripts

\begin{Shaded}
\begin{Highlighting}[]
\KeywordTok{library}\NormalTok{(MODIStsp) }
\CommentTok{# --> Specify the path to a valid options file saved in advance from MODIStsp GUI }
\NormalTok{options_file <-}\StringTok{ "X:/yourpath/youroptions.json"} 
  
\CommentTok{# --> Launch the processing}
\KeywordTok{MODIStsp}\NormalTok{(}\DataTypeTok{gui =} \OtherTok{FALSE}\NormalTok{, }\DataTypeTok{options_file =}\NormalTok{ options_file)}
\end{Highlighting}
\end{Shaded}

\subsection{Looping on different spatial
extents}\label{looping-on-different-spatial-extents}

Specifying also the \texttt{spatial\_file\_path\_} parameter overrides
for example the output extent of the selected Options File. This allows
to perform the same preprocessing on different extents using a single
Options File, by looping on an array of spatial files representing the
desired output extents. For example:

\begin{Shaded}
\begin{Highlighting}[]
\KeywordTok{library}\NormalTok{(MODIStsp) }
\CommentTok{# --> Specify the path to a valid options file saved in advance from MODIStsp GUI }
\NormalTok{options_file <-}\StringTok{ "X:/yourpath/youroptions.json"} 

\CommentTok{# --> Create a character array containing a list of shapefiles (or other spatial files)}
\NormalTok{extent_list <-}\StringTok{ }\KeywordTok{list.files}\NormalTok{(}\StringTok{"X:/path/containing/some/shapefiles/"}\NormalTok{, }\StringTok{"}\CharTok{\textbackslash{}\textbackslash{}}\StringTok{.shp$"}\NormalTok{)  }

\CommentTok{# --> Loop on the list of spatial files and run MODIStsp using each of them to automatically }
\CommentTok{# define the output extent (A separate output folder is created for each input spatial file).}

\ControlFlowTok{for}\NormalTok{ (single_shape }\ControlFlowTok{in}\NormalTok{ extent_list) \{}
  \KeywordTok{MODIStsp}\NormalTok{(}\DataTypeTok{gui =} \OtherTok{FALSE}\NormalTok{, }\DataTypeTok{options_file =}\NormalTok{ options_file, }\DataTypeTok{spatial_file_path =}\NormalTok{ single_shape )}
\NormalTok{\}}
\end{Highlighting}
\end{Shaded}

\section{Stand-alone execution and scheduled
processing}\label{stand-alone-execution-and-scheduled-processing}

\subsection{Stand-alone execution}\label{stand-alone-execution}

\begin{itemize}
\tightlist
\item
  MODIStsp can be also executed as a ``stand-alone'' application(i.e.,
  without having to open R/RStudio); to do this, from R launch the
  function \texttt{MODIStsp\_install\_launcher()}.
\end{itemize}

In a Linux operating system this function creates a desktop entry
(accessible from the menu in the sections ``Science'' and ``Geography'')
and a symbolic link in a known path (default: /usr/bin/MODIStsp). In
Windows, a link in the Start Menu and optionally a desktop shortcut are
created. See \texttt{?install\_MODIStsp\_launcher} for details and path
customization.

Double-clicking those files or launching them from a shell without
parameters will launch \texttt{MODIStsp} in interactive mode.
Non-interactive mode is triggered by adding the ``-g'' argument to the
call, and specifying the path to a valid Options File as ``-s''
argument:

\begin{itemize}
\item
  Linux: \texttt{MODIStsp\ -g\ -s\ "/yourpath/youroptions.json"} (see
  \texttt{MODIStsp\ -h} for details).
\item
  Windows:\texttt{your\_r\_library\textbackslash{}MODIStsp\textbackslash{}ExtData\textbackslash{}Launcher\textbackslash{}MODIStsp.bat\ -g\ -s\ "yourpath/youroptions.json"}
  (see
  \texttt{C:\textbackslash{}Users\textbackslash{}you\textbackslash{}Desktop\textbackslash{}MODIStsp\ -h}
  for details).
\end{itemize}

If you do not want to install any link, launchers can be found in the
subdirectory ``MODIStsp/ExtData/Launcher'' of your library path.

\subsection{Scheduled Processing}\label{scheduled-processing}

Stand-alone non-interactive execution can be used to periodically and
automatically update the time series of a selected product over a given
study area. To do that, you should simply:

\begin{enumerate}
\def\labelenumi{\arabic{enumi}.}
\item
  Open the \texttt{MODIStsp} GUI, define the parameters of the
  processing specifying a date in the future as the ``Ending Date'' and
  save the processing options. Then quit the program.
\item
  Schedule non-interactive execution of the launcher installed as seen
  before (or located in the subdirectory ``MODIStsp/ExtData/Launcher''
  of your library path) as windows scheduled task (or linux ``cron''
  job) according to a specified time schedule, specifying the path of a
  previously saved Options file as additional argument:
\end{enumerate}

\paragraph{On Linux}\label{on-linux}

\begin{itemize}
\tightlist
\item
  edit your crontab by opening a terminal and type:
\end{itemize}

\begin{Shaded}
\begin{Highlighting}[]
  \ExtensionTok{crontab}\NormalTok{ -e}
\end{Highlighting}
\end{Shaded}

\begin{itemize}
\tightlist
\item
  add an entry for the launcher. For example, if you have installed it
  in /usr/bin and you want to run the tool every day at 23.00, add the
  following row:
\end{itemize}

\begin{Shaded}
\begin{Highlighting}[]
  \ExtensionTok{0}\NormalTok{ 23 * * * /bin/bash /usr/bin/MODIStsp -g -s }\StringTok{"/yourpath/youroptions.json"}
\end{Highlighting}
\end{Shaded}

\paragraph{On Windows}\label{on-windows-1}

\begin{itemize}
\tightlist
\item
  create a Task following
  \href{https://technet.microsoft.com/en-us/library/cc748993.aspx}{these
  instructions}; add the path of the MODIStsp.bat launcher as Action
  (point 6), and specifying
  \texttt{-g\ -s\ "X:/yourpath/youroptions.json"} as argument.
\end{itemize}

\section{Outputs Format and Naming
Conventions}\label{outputs-format-and-naming-conventions}

\subsection{Single-band outputs}\label{single-band-outputs}

Output raster files are saved in specific subfolders of the main output
folder. In particular, \textbf{a separate subfolder} is created for each
processed original MODIS layer, Quality Indicator or Spectral Index.
Each subfolder contains one image for each processed date, created
according to the following naming conventions:

\begin{verbatim}
myoutfolder/"Layer"/"ProdCode"_"Layer"_"YYYY"_"DOY"."ext"
\end{verbatim}

 \emph{(e.g., myoutfolder/NDVI/MOD13Q1\_NDVI\_2000\_065.dat)}

, where:

\begin{itemize}
\tightlist
\item
  \textbf{\emph{Layer}} is a short name describing the dataset (e.g.,
  b1\_Red, NDII, UI);
\item
  \textbf{\emph{ProdCode}} is the code name of the MODIS product from
  which the image was derived (e.g., MOD13Q1);
\item
  \textbf{\emph{YYYY}} and \textbf{\emph{DOY}} correspond to the year
  and DOY (Day of the Year) of acquisition of the original MODIS image;
\item
  \textbf{\emph{ext}} is the file extension (.tif for GTiff outputs, or
  .dat for ENVI outputs).
\end{itemize}

\begin{center}\rule{0.5\linewidth}{\linethickness}\end{center}

\subsection{Virtual multi-band
outputs}\label{virtual-multi-band-outputs}

ENVI and/or GDAL virtual time series files and \emph{RasterStack} RData
objects are instead stored \textbf{in the ``Time\_Series'' subfolder} if
required.

Naming convention for these files is as follow:

\begin{verbatim}
myoutfolder/Time_Series/"vrt_type"/"Sensor"/"Layer"/"ProdCode"_"Layer"_"StartDOY"_"StartYear_"EndDOY"_"EndYear_"suffix".ext" 
\end{verbatim}

 \emph{(e.g.,
myoutfolder/Time\_Series/RData/Terra/NDVI/MOD13Q1\_MYD13Q1\_NDVI\_49\_2000\_353\_2015\_RData.RData)}

, where:

\begin{itemize}
\tightlist
\item
  \textbf{\emph{vrt\_type}} indicates the type of virtual file
  (``RData'', ``GDAL'' or ``ENVI\_META'');
\item
  \textbf{\emph{Sensor}} indicates to which MODIS sensor the time series
  belongs (``Terra'', ``Aqua'', ``Mixed'' or ``Combined'' (for MCD*
  products));
\item
  \textbf{\emph{Layer}} is a short name describing the dataset (e.g.,
  b1\_Red, NDII, UI);
\item
  \textbf{\emph{ProdCode}} is the code name of the MODIS product from
  which the image was derived (e.g., MOD13Q1);
\item
  \textbf{\emph{StartDOY}}, \textbf{\emph{StartYear}},
  \textbf{\emph{EndDOY}} and \textbf{\emph{EndYear}} indicate the
  temporal extent of the time serie created;
\item
  \textbf{\emph{suffix}} indicates the type of virtual file (ENVI, GDAL
  or RData);
\item
  \textbf{\emph{ext}} is the file extension (``.vrt'' for gdal virtual
  files, ``META'' for ENVI meta files or ``RData'' for \texttt{R} raster
  stacks).
\end{itemize}

\section{Accessing the processed time series from
R}\label{accessing-the-processed-time-series-from-r}

Preprocessed MODIS data can be retrieved within \texttt{R} either by
accessing the single-date raster files, or by loading the saved
\emph{RasterStack} objects.

Any single-date image can be accessed by simply opening it with a
\texttt{raster} command:

\begin{Shaded}
\begin{Highlighting}[]
\KeywordTok{library}\NormalTok{(raster)}
\NormalTok{modistsp_file <-}\StringTok{ "/my_outfolder/EVI/MOD13Q1_2005_137_EVI.tif"}
\NormalTok{my_raster <-}\StringTok{ }\KeywordTok{raster}\NormalTok{(modistsp_file)}
\end{Highlighting}
\end{Shaded}

\texttt{rasterStack} time series containing all the processed data for a
given parameter (saved in the ``Time Series/RData'' subfolder - see
\href{output.html}{here} for details) can be opened by:

\begin{Shaded}
\begin{Highlighting}[]
\NormalTok{in_virtual_file <-}\StringTok{ "/my_outfolder/Time_Series/RData/Terra/EVI/MOD13Q1_MYD13Q1_NDVI_49_2000_353_2015_RData.RData"} 
\NormalTok{indata          <-}\StringTok{ }\KeywordTok{get}\NormalTok{(}\KeywordTok{load}\NormalTok{(in_virtual_file))}
\end{Highlighting}
\end{Shaded}

This second option allows accessing the complete data stack and
analyzing it using the functionalities for raster/raster time series
analysis, extraction and plotting provided for example by the
\texttt{raster} or \texttt{rasterVis} packages.

\subsection{Extracting Time Series Data on Areas of
Interest}\label{extracting-time-series-data-on-areas-of-interest}

\texttt{MODIStsp} provides an efficient function
(\texttt{MODIStsp\textbackslash{}\_extract}) for extracting time series
data at specific locations. The function takes as input a
\emph{RasterStack} virtual object created by \texttt{MODIStsp} (see
above), the starting and ending dates for the extraction and a standard
\_Sp*\_ object (or an ESRI shapefile name) specifying the locations
(points, lines or polygons) of interest, and provides as output a
\texttt{xts} object or \texttt{data.frame} containing time series data
for those locations.

If the input is of class \emph{SpatialPoints}, the output object
contains one column for each point specified, and one row for each date.
If it is of class \emph{SpatialPolygons} (or \emph{SpatialLines}), it
contains one column for each polygon (or each line), with values
obtained applying the function specified as the ``FUN'' argument (e.g.,
mean, standard deviation, etc.) on pixels belonging to the polygon (or
touched by the line), and one row for each date.

As an example the following code:

\begin{Shaded}
\begin{Highlighting}[]
  \CommentTok{#Set the input paths to raster and shape file}
\NormalTok{  infile    <-}\StringTok{ 'myoutfolder/Time_Series/RData/Mixed/MOD13Q1_MYD13Q1_NDVI_49_2000_353_2015_RData.RData'}  
\NormalTok{  shpname   <-}\StringTok{ 'path_to_file/rois.shp'}  
  \CommentTok{#Set the start/end dates for extraction}
\NormalTok{  startdate <-}\StringTok{ }\KeywordTok{as.Date}\NormalTok{(}\StringTok{"2010-01-01"}\NormalTok{)  }
\NormalTok{  enddate   <-}\StringTok{ }\KeywordTok{as.Date}\NormalTok{(}\StringTok{"2014-12-31"}\NormalTok{)    }
  \CommentTok{#Load the RasterStack}
\NormalTok{  inrts     <-}\StringTok{ }\KeywordTok{get}\NormalTok{(}\KeywordTok{load}\NormalTok{(infile)) }
  \CommentTok{# Compute average and St.dev}
\NormalTok{  dataavg   <-}\StringTok{ }\KeywordTok{MODIStsp_extract}\NormalTok{(inrts, shpname, startdate, enddate, }\DataTypeTok{FUN =} \StringTok{'mean'}\NormalTok{, }\DataTypeTok{na.rm =}\NormalTok{ T)}
\NormalTok{  datasd    <-}\StringTok{ }\KeywordTok{MODIStsp_extract}\NormalTok{ (inrts, shpname, startdate, enddate, }\DataTypeTok{FUN =} \StringTok{'sd'}\NormalTok{, }\DataTypeTok{na.rm =}\NormalTok{ T)}
  \CommentTok{# Plot average time series for the polygons}
  \KeywordTok{plot.xts}\NormalTok{(dataavg) }
\end{Highlighting}
\end{Shaded}

loads a \emph{RasterStack} object containing 8-days 250 m resolution
time series for the 2000-2015 period and extracts time series of average
and standard deviation values over the different polygons of a user's
selected shapefile on the 2010-2014 period.

\section{Problems and Issues}\label{problems-and-issues}

Solutions to some common \textbf{installation and processing problems}
can be found in MODIStsp FAQ:

\url{http://lbusett.github.io/MODIStsp/articles/faq.html}

\begin{itemize}
\tightlist
\item
  Please \textbf{report any issues} you may encounter in our issues page
  on GitHub:
\end{itemize}

\url{https://github.com/lbusett/MODIStsp/issues}

\section{Citation}\label{citation}

To cite MODIStsp please use:

L. Busetto, L. Ranghetti (2016) MODIStsp: An R package for automatic
preprocessing of MODIS Land Products time series, Computers \&
Geosciences, Volume 97, Pages 40-48, ISSN 0098-3004,
\url{http://dx.doi.org/10.1016/j.cageo.2016.08.020}, URL:
\url{https://github.com/lbusett/MODIStsp}.

\hypertarget{installing-r-and-gdal}{\section{Installing R and
GDAL}\label{installing-r-and-gdal}}

\subsection{Installing R}\label{installing-r}

\subsubsection{Windows}\label{windows}

Download and install the latest version of R which can be found
\href{https://cran.r-project.org/bin/windows/base}{here}.

\subsubsection{Linux}\label{linux}

Please refer to the documentation which can be found
\href{https://cran.r-project.org/bin/linux}{here}, opening the directory
relative to your Linux distribution. The documentation provides
instruction to add CRAN repositories and to install the latest R
version. With Ubuntu 15.10 Wily (and newer) this step is not mandatory
(although recommended), since packaged version of R is \textgreater{}=
3.2.1 (although not the latest); in this case, user can install R by
simply typing in a terminal

\begin{Shaded}
\begin{Highlighting}[]
\FunctionTok{sudo}\NormalTok{ apt-get install r-base}
\end{Highlighting}
\end{Shaded}

\subsection{Installing GDAL \textgreater{}=
1.11.1}\label{installing-gdal-1.11.1}

\subsubsection{Windows}\label{windows-1}

The easiest way to install GDAL on Windows is from the
\href{https://trac.osgeo.org/osgeo4w/}{OSGeo4W Website}

\begin{enumerate}
\def\labelenumi{\arabic{enumi}.}
\tightlist
\item
  Open the \href{https://trac.osgeo.org/osgeo4w/}{OSGeo4W Website}
\item
  In the \textbf{Quick Start for OSGeo4W Users} section, select the
  download of 32bit or 64bit of OSGeo4W network installer
\item
  Run the installer
\end{enumerate}

\begin{itemize}
\tightlist
\item
  \emph{Easiest Option}:

  \begin{itemize}
  \tightlist
  \item
    Select \textbf{Express Desktop Install}, then proceed with the
    installation. This will install GDAL and also other useful Spatial
    Processing software like \href{http://www.qgis.org/}{QGIS} and
    \href{https://grass.osgeo.org/}{GRASS GIS}
  \end{itemize}
\item
  \emph{Advanced Option}:

  \begin{itemize}
  \tightlist
  \item
    Select \textbf{Advanced Install}, then click on ``Next'' a few times
    until you reach the ``Select Packages'' screen.
  \item
    Click on ``Commandline\_Utilities\_'', and on the list look for
    ``\_gdal: The GDAL/OGR library\ldots{}" entry
  \item
    Click on ``Skip'': the word ``skip'' will be replaced by the current
    GDAL version number
  \item
    Click on ``Next'' a few times to install GDAL
  \end{itemize}
\end{itemize}

\subsubsection{Debian and Ubuntu-based
systems}\label{debian-and-ubuntu-based-systems}

\begin{enumerate}
\def\labelenumi{\arabic{enumi}.}
\item
  Ensure that your repositories contain a version of \texttt{gdal-bin}
  \textgreater{}= 1.11.1. In particular, official repositories of Ubuntu
  15.04 Vivid (or older) and Debian Jessie (or older) provide older
  versions of GDAL, so it is necessary to add UbuntuGIS-unstable
  repository before installing. To do this, follow instructions
  \href{https://launchpad.net/~ubuntugis/+archive/ubuntu/ubuntugis-unstable}{here}).
  With Ubuntu 15.10 Wily (and newer) this step is not mandatory,
  although recommended in order to have updated version of GDAL
  installed.
\item
  To install GDAL, open a terminal and type

\begin{Shaded}
\begin{Highlighting}[]
\FunctionTok{sudo}\NormalTok{ apt-get install gdal-bin}
\end{Highlighting}
\end{Shaded}
\end{enumerate}

\subsubsection{ArchLinux}\label{archlinux}

GDAL is maintained updated to the latest version as binary package
within the community repository; although that, the support for HDF4
format is not included. To bypass this problem, ArchLinux users can
install \texttt{gdal-hdf4} package from AUR (see
\href{https://wiki.archlinux.org/index.php/Arch_User_Repository\#Installing_packages}{here}
or \href{https://archlinux.fr/yaourt-en}{here} for the package
installation from AUR). This package is updated manually after each
release of \texttt{gdal} on the community repository, so a temporal
shift between a new \texttt{gdal} release and the update of
\texttt{gdal-hdf4} could happen. If you want to manually add the support
for HDF4 in case \texttt{gdal-hdf4} is out-of-date, you can do it
following \href{https://notehub.org/fctdn}{these instructions}.

\subsubsection{Other Linux systems}\label{other-linux-systems}

Install the packaged binary of GDAL included in your specific
distribution; if the version is older than 1.11.1, or if the support for
HDF4 format is not included, you can manually install the HDF4 library
and compile the source code by adding the parameter
\texttt{-\/-with-hdf4} to the \texttt{configure} instruction).

\section*{References}\label{references}
\addcontentsline{toc}{section}{References}

\hypertarget{refs}{}
\hypertarget{ref-Boschetti2014}{}
Boschetti, M., F. Nutini, G. Manfron, P. A. Brivio, and A. Nelson. 2014.
``Comparative analysis of normalised difference spectral indices derived
from MODIS for detecting surface water in flooded rice cropping
systems.'' \emph{PLoS ONE} 9 (2).
doi:\href{https://doi.org/10.1371/journal.pone.0088741}{10.1371/journal.pone.0088741}.

\hypertarget{ref-Gamon1999}{}
Gamon, J.A., and J.S. Surfus. 1999. ``Assessing leaf pigment content and
activity with a reflectometer.'' \emph{New Phytologis} 143 (1): 105--17.

\hypertarget{ref-Gitelson1998}{}
Gitelson, A.A., and M.N. Merzlyak. 1998. ``Remote sensing of chlorophyll
concentration in higher plant leaves.'' \emph{Advances in Space
Research} 22 (5): 689--92.
doi:\href{https://doi.org/10.1016/S0273-1177(97)01133-2}{10.1016/S0273-1177(97)01133-2}.

\hypertarget{ref-Hall2002}{}
Hall, Dorothy K, George A Riggs, Vincent V Salomonson, Nicolo E
DiGirolamo, and Klaus J Bayr. 2002. ``MODIS snow-cover products.''
\emph{Remote Sensing of Environment} 83 (1-2): 181--94.
doi:\href{https://doi.org/10.1016/S0034-4257(02)00095-0}{10.1016/S0034-4257(02)00095-0}.

\hypertarget{ref-Hengl2010}{}
Hengl, T. 2010. ``Download and resampling of MODIS images.''
\url{http://spatial-analyst.net/wiki/index.php?title=Download_and_resampling_of_MODIS_images}.

\hypertarget{ref-Huete2002}{}
Huete, A., K. Didan, T. Miura, E.P. Rodriguez, X. Gao, and L.G.
Ferreira. 2002. ``Overview of the radiometric and biophysical
performance of the MODIS vegetation indices.'' \emph{Remote Sensing of
Environment} 83 (1-2): 195--213.
doi:\href{https://doi.org/10.1016/S0034-4257(02)00096-2}{10.1016/S0034-4257(02)00096-2}.

\hypertarget{ref-Huete1988}{}
Huete, A.R. 1988. ``A soil-adjusted vegetation index (SAVI).''
\emph{Remote Sensing of Environment} 25 (3): 295--309.
doi:\href{https://doi.org/10.1016/0034-4257(88)90106-X}{10.1016/0034-4257(88)90106-X}.

\hypertarget{ref-HUNTJR1989}{}
Hunt, J.R., and B. Rock. 1989. ``Detection of changes in leaf water
content using Near- and Middle-Infrared reflectances.'' \emph{Remote
Sensing of Environment} 30 (1): 43--54.
doi:\href{https://doi.org/10.1016/0034-4257(89)90046-1}{10.1016/0034-4257(89)90046-1}.

\hypertarget{ref-Naimi2014}{}
Naimi, B. 2014. ``ModisDownload: an R function to download, mosaic, and
reproject the MODIS images.'' \url{http://r-gis.net/?q=ModisDownload}.

\hypertarget{ref-Rouse1973}{}
Rouse, J.W.J., R.H. Haas, J.A. Schell, and D.W. Deering. 1973.
``Monitoring vegetation systems in the Great Plains with ERTS. Third
ERTS Symposium, NASA SP-351. U.S. Gov. Printing office.'' Edited by S.C.
Freden, E.P. Mercanti, and M.A. Becker. NASA.

\hypertarget{ref-Tucker1979}{}
Tucker, C.J. 1979. ``Red and photographic infrared linear combinations
for monitoring vegetation.'' \emph{Remote Sensing of Environment} 8 (2):
127--50.
doi:\href{https://doi.org/10.1016/0034-4257(79)90013-0}{10.1016/0034-4257(79)90013-0}.


\end{document}
